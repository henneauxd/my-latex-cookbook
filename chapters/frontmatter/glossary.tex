% Use built-in macro \glossaryname for proper internationalization. With polyglossia, it
% will contain \text<language>{<glossary translation>}, which has been taken care of
% using \pdfstringdefDisableCommands{} in the class file

% \addchap{\glossaryname}%
% \label{ch:glossary}

% \emph{%
%     \TransGlossaryLegend{}%
% }%

% % Print "unsorted" glossaries; these are in fact sorted, but externally using bib2gls.
% % These will throw 'Token not allowed in PDF, removing \text<language>' warning.
% % Specify title= manually if that gets too annoying.
% \printunsrtglossary[
%     type=symbols,
%     style=symbunitlong,
% ]
% \printunsrtglossary[
%     type=numbers,
%     style=numberlong,
% ]
% \printunsrtglossary[
%     type=subscripts,
%     style=mcolalttree,
%     nonumberlist,
% ]
% \printunsrtglossary[
%     type=abbreviations,
%     style=long3colheader,
%     title=\TransAcronyms{},
% ]

\addchap{Nomenclature}% glossaryname

\emph{
    Description
}

% For all sorts of styles, see also
% https://www.dickimaw-books.com/gallery/glossaries-styles/

% Simply pass the `nonumberlist` parameter where desired/required:
\printunsrtglossary[
    type=symbols,
    style=symbunitlong,
    nonumberlist,
]
\printunsrtglossary[
    type=numbers,
    style=numberlong,
    nonumberlist,
]
\printunsrtglossary[
    type=subscripts,
    style=mcolalttree,
    nonumberlist,
]
\printunsrtglossary[
    type=abbreviations,
    % If `nonumberlist` is passed, the `long3colheader` style simply leaves the
    % corresponding table cells *empty* (leading to an entirely empty column), but does
    % not actually remove the column. So use a different, but equivalent style
    % altogether:
    style=longheader,
    % The `longheader` style prints the page list behind the description, just not in a
    % separate column. So also explicitly suppress the generation of that:
    nonumberlist,
    title=\TransAcronyms{},
]